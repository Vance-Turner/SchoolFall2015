\documentclass[11pt,letterpaper]{article}
\usepackage[utf8]{inputenc}
\usepackage{amsmath}
\usepackage{amsfonts}
\usepackage{amssymb}
\usepackage[letterpaper, portrait, margin=0.5in]{geometry}
\linespread{2}
\usepackage{times}
\author{Vance Turnewitsch}
\begin{document}
Find the general solution of: $x'=x(1-x)-h$\\

We expand into: $x'=x-x^2-h$ and rearrange: $dx/dt=-x^2+x-h$. Now we manipulate to get:

$1/(-x^2+x-h)dx = dt \Longrightarrow 1/(x^2-x+h)=-dt$ and we integrate: $\int(1/(x^2-x+h))dx=-\int dt$

And we get: $\int(1/(x^2-x+h))dx=c-t$ where $c$ is some general constant.

Now we consider the denominator in the left integral: $x^2-x+h$ which we observe has two solutions: $x={1\pm\sqrt{1-4h}\over 2}$ 

Thus we can factor the fraction as follows: ${1\over x^2-x+h}={1\over (x+a)(x+b)}$ where $a={1+\sqrt{1-4h}\over 2}$ and $b={1-\sqrt{1-4h}\over 2}$ assuming that both $a,b$ are real here!


We now have: $\int{1\over (x+a)(x+b)}dx=c-t$ We now proceed to use partial fractions on the left integrand:

${1\over (x+a)(x+b)}=A/(x+a) + B/(x+b)$ and expanding we get:

$1 = A(x+b) + B(x+a) \Longrightarrow 1=Ax+Ab + Bx+Ba \Longrightarrow 1=x(A+B) + Ab + Ba$

We observe this can be represented as a matrix equation:

$\left[\begin{array}{ccc}
x & x \\
b & a \\
\end{array}\right]
\left[\begin{array}{c}
A\\
B\\\end{array}\right]=
\left[\begin{array}{c}
0 \\
1 \\
\end{array}\right]
$  and we multiply by the inverse to get:


$\left[\begin{array}{c}
A\\
B\\\end{array}\right]={1\over x(a-b)}
\left[\begin{array}{cc}
a & -x \\
-b & x \\
\end{array}\right]
\left[\begin{array}{c}
0 \\
1 \\
\end{array}\right]=
{1\over x(a-b)}
\left[\begin{array}{c}
-x \\
x \\
\end{array}\right]=
{1\over a-b}
\left[\begin{array}{c}
-1 \\
1 \\
\end{array}\right]
$

We thus find that: $A=-1/(a-b)$ and $B=1/(a-b)$

Our left integral is now: $\int {-1/(a-b)\over(x+a)}+{1/(a-b)\over x+b}dx={-1\over a-b}\int {1\over x+a}dx+{1\over a-b}\int{1\over x+b}dx$ and after integrating we get:

${-1\over a-b}\ln{|x+a|}+{1\over a-b}\ln{|x+b|}=c_2-t$ where we have absorbed the constant from the left integration into the one on the right to create $c_2$.

We now do some simplifying:

$\ln|x+b|-\ln|x+a|=(a-b)(c_2-t) \Longrightarrow \ln|{x+b\over x+a}|=(a-b)(c_2-t)$ And we place both left hand side and right hand side in exponent of base $e$ to get:

$|{x+b\over x+a}|=e^{(a-b)(c_2-t)}$ and we solve to get:

${x+b\over x+a}=\pm e^{(a-b)(c_2-t)} \Longrightarrow {x+b\over x+a}=d e^{(a-b)(c_2-t)}$ where we have used $d$ to absorb that irritating $\pm$

$x+b=(x+a)de^{(a-b)(c_2-t)}=xde^{(a-b)(c_2-t)}+ade^{(a-b)(c_2-t)}$

$x-xde^{(a-b)(c_2-t)}=ade^{(a-b)(c_2-t)}-b$

$x(1-de^{(a-b)(c_2-t)})=ade^{(a-b)(c_2-t)}-b$

$x={ade^{(a-b)(c_2-t)}-b\over 1-de^{(a-b)(c_2-t)}}$

Work on the constants

$a-b={1+\sqrt{1-4h}\over 2}-{1-\sqrt{1-4h}\over 2}={1-1+\sqrt{1-4h}+\sqrt{1-4h}\over 2}=\sqrt{1-4h}$

We also observe that: $b=a-\sqrt{1-4h}$

Thus plugging these in we find:

$x(t)={{(1+\sqrt{1-4h})\over 2}de^{\sqrt{1-4h}(c_2-t)}-{1+\sqrt{1-4h}\over 2}+\sqrt{1-4h}\over 1-de^{\sqrt{1-4h}(c_2-t)}}$ and after simplifying the fraction we find:

$x(t)=[(1+\sqrt{1-4h})de^{\sqrt{1-4h}(c_2-t)}-1+\sqrt{1-4h}]/[ 2(1-de^{\sqrt{1-4h}(c_2-t)})]$

Work on when we have imaginary roots...

We now proceed to work on the equation when: $1-4h<0$

We thus have two imaginary solutions: $x={1\pm i\sqrt{1-4h}\over 2}$ which we shall again refer to as: $a,b$. 

We know though that there exists $c,d$ such that: $a=c+di$ and $b=c-di$ Our fraction in the left-hand integral thus becomes:

${1\over (x+c+di)(x+c-di)}={1\over x^2+xc-xdi+xc+c^2-cdi+xdi+cdi+d^2}=1/(x^2+2xc+c^2+d^2)=1/((x+c)^2+d^2)$

We thus now have this situation:

$\int 1/((x+c)^2+d^2)dx = f-t$ where we have chosen $f$ to be our constant from the right-hand integration.

We use a CAS to integrate the left-hand side.

$\arctan({(x+c)/d})/d=g-t$ where we have absorbed the costant from left hand integration into $f$

$\arctan((x+c)/d)=d(g-t)$

$(x+c)/d=\tan(d(g-t))$

$x+c=d\tan(d(g-t))$

$x=d\tan(d(g-t))-c$ or in expanded form: ;)

$x(t)=(\sqrt{1-4h}/2)\tan[(\sqrt{1-4h}/2)(g-t)]-1/2$ is our second final solution.

Thus the general solution is:

$x(t)=\begin{cases}
[(1+\sqrt{1-4h})de^{\sqrt{1-4h}(c_2-t)}-1]/[ 2(1-de^{\sqrt{1-4h}(c_2-t)})] \textrm{ when } 1-4h>0 \\
(\sqrt{1-4h}/2)\tan[(\sqrt{1-4h}/2)(g-t)]-1/2 \textrm{ when } 1-4h<0 \\
\end{cases}$

where $d=\pm 1$ and ${c_2,g,h}$ are constants to be determined.
\end{document}